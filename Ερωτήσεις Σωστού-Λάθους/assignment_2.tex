%%%%%%%%%%%%%%%%%%%%%%%%%%%%%%%%%%%%%%%%%
% Programming/Coding Assignment
% LaTeX Template
%
% This template has been downloaded from:
% http://www.latextemplates.com
%
% Original author:
% Ted Pavlic (http://www.tedpavlic.com)
%
% Note:
% The \lipsum[#] commands throughout this template generate dummy text
% to fill the template out. These commands should all be removed when 
% writing assignment content.
%
% This template uses a Perl script as an example snippet of code, most other
% languages are also usable. Configure them in the "CODE INCLUSION 
% CONFIGURATION" section.
%
%%%%%%%%%%%%%%%%%%%%%%%%%%%%%%%%%%%%%%%%%

%----------------------------------------------------------------------------------------
%	PACKAGES AND OTHER DOCUMENT CONFIGURATIONS
%----------------------------------------------------------------------------------------

\documentclass{article}

\usepackage{fancyhdr} % Required for custom headers
\usepackage{lastpage} % Required to determine the last page for the footer
\usepackage{extramarks} % Required for headers and footers
\usepackage[usenames,dvipsnames]{color} % Required for custom colors
\usepackage{graphicx} % Required to insert images
\usepackage{listings} % Required for insertion of code
\usepackage{courier} % Required for the courier font
\usepackage{lipsum} % Used for inserting dummy 'Lorem ipsum' text into the template
\usepackage{fontspec}
\usepackage{float}
\setmainfont{PF Universal}

% Margins
\topmargin=-0.45in
\evensidemargin=0in
\oddsidemargin=0in
\textwidth=6.5in
\textheight=9.0in
\headsep=0.25in

\linespread{1.1} % Line spacing

% Set up the header and footer
\pagestyle{fancy}
\lhead{\hmwkAuthorName} % Top left header
\chead{\hmwkClass\ (\hmwkClassInstructor\ \hmwkClassTime): \hmwkTitle} % Top center head
\rhead{\firstxmark} % Top right header
\lfoot{\lastxmark} % Bottom left footer
\cfoot{} % Bottom center footer
\rfoot{Page\ \thepage\ of\ \protect\pageref{LastPage}} % Bottom right footer
\renewcommand\headrulewidth{0.4pt} % Size of the header rule
\renewcommand\footrulewidth{0.4pt} % Size of the footer rule

\setlength\parindent{0pt} % Removes all indentation from paragraphs

%----------------------------------------------------------------------------------------
%	CODE INCLUSION CONFIGURATION
%----------------------------------------------------------------------------------------

\definecolor{MyDarkGreen}{rgb}{0.0,0.4,0.0} % This is the color used for comments
\lstloadlanguages{Perl} % Load Perl syntax for listings, for a list of other languages supported see: ftp://ftp.tex.ac.uk/tex-archive/macros/latex/contrib/listings/listings.pdf
\lstset{language=Perl, % Use Perl in this example
        frame=single, % Single frame around code
        basicstyle=\small\ttfamily, % Use small true type font
        keywordstyle=[1]\color{Blue}\bf, % Perl functions bold and blue
        keywordstyle=[2]\color{Purple}, % Perl function arguments purple
        keywordstyle=[3]\color{Blue}\underbar, % Custom functions underlined and blue
        identifierstyle=, % Nothing special about identifiers                                         
        commentstyle=\usefont{T1}{pcr}{m}{sl}\color{MyDarkGreen}\small, % Comments small dark green courier font
        stringstyle=\color{Purple}, % Strings are purple
        showstringspaces=false, % Don't put marks in string spaces
        tabsize=5, % 5 spaces per tab
        %
        % Put standard Perl functions not included in the default language here
        morekeywords={rand},
        %
        % Put Perl function parameters here
        morekeywords=[2]{on, off, interp},
        %
        % Put user defined functions here
        morekeywords=[3]{test},
       	%
        morecomment=[l][\color{Blue}]{...}, % Line continuation (...) like blue comment
        numbers=left, % Line numbers on left
        firstnumber=1, % Line numbers start with line 1
        numberstyle=\tiny\color{Blue}, % Line numbers are blue and small
        stepnumber=5 % Line numbers go in steps of 5
}

% Creates a new command to include a perl script, the first parameter is the filename of the script (without .pl), the second parameter is the caption
\newcommand{\perlscript}[2]{
\begin{itemize}
\item[]\lstinputlisting[caption=#2,label=#1]{#1.pl}
\end{itemize}
}

%----------------------------------------------------------------------------------------
%	DOCUMENT STRUCTURE COMMANDS
%	Skip this unless you know what you're doing
%----------------------------------------------------------------------------------------

% Header and footer for when a page split occurs within a problem environment
\newcommand{\enterProblemHeader}[1]{
\nobreak\extramarks{#1}{#1 continued on next page\ldots}\nobreak
\nobreak\extramarks{#1 (continued)}{#1 continued on next page\ldots}\nobreak
}

% Header and footer for when a page split occurs between problem environments
\newcommand{\exitProblemHeader}[1]{
\nobreak\extramarks{#1 (continued)}{#1 continued on next page\ldots}\nobreak
\nobreak\extramarks{#1}{}\nobreak
}

\setcounter{secnumdepth}{0} % Removes default section numbers
\newcounter{homeworkProblemCounter} % Creates a counter to keep track of the number of problems

\newcommand{\homeworkProblemName}{}
\newenvironment{homeworkProblem}[1][Πρόβλημα \arabic{homeworkProblemCounter}]{ % Makes a new environment called homeworkProblem which takes 1 argument (custom name) but the default is "Problem #"
\stepcounter{homeworkProblemCounter} % Increase counter for number of problems
\renewcommand{\homeworkProblemName}{#1} % Assign \homeworkProblemName the name of the problem
\section{\homeworkProblemName} % Make a section in the document with the custom problem count
\enterProblemHeader{\homeworkProblemName} % Header and footer within the environment
}{
\exitProblemHeader{\homeworkProblemName} % Header and footer after the environment
}

\newcommand{\problemAnswer}[1]{ % Defines the problem answer command with the content as the only argument
\noindent\framebox[\columnwidth][c]{\begin{minipage}{0.98\columnwidth}#1\end{minipage}} % Makes the box around the problem answer and puts the content inside
}

\newcommand{\homeworkSectionName}{}
\newenvironment{homeworkSection}[1]{ % New environment for sections within homework problems, takes 1 argument - the name of the section
\renewcommand{\homeworkSectionName}{#1} % Assign \homeworkSectionName to the name of the section from the environment argument
\subsection{\homeworkSectionName} % Make a subsection with the custom name of the subsection
\enterProblemHeader{} % Header and footer within the environment
}{
\enterProblemHeader{} % Header and footer after the environment
}

%----------------------------------------------------------------------------------------
%	NAME AND CLASS SECTION
%----------------------------------------------------------------------------------------

\newcommand{\hmwkTitle}{Ερωτήσεις Σωστού-Λάθους\ \#5} % Assignment title
\newcommand{\hmwkDueDate}{Σάββατο,\ Σεπτέμβριος\ 28,\ 2013} % Due date
\newcommand{\hmwkClass}{ΑΝΕΠ} % Course/class
\newcommand{\hmwkClassTime}{18:00} % Class/lecture time
\newcommand{\hmwkClassInstructor}{Θεοφίλης} % Teacher/lecturer
\newcommand{\hmwkAuthorName}{Γεώργιος Θεοφίλης} % Your name

%----------------------------------------------------------------------------------------
%	TITLE PAGE
%----------------------------------------------------------------------------------------

\title{
\vspace{2in}
\textmd{\textbf{\hmwkClass:\ \hmwkTitle}}\\
\normalsize\vspace{0.1in}\small{Due\ on\ \hmwkDueDate}\\
\vspace{0.1in}\large{\textit{\hmwkClassInstructor\ \hmwkClassTime}}
\vspace{3in}
}

\author{\textbf{\hmwkAuthorName}}
\date{} % Insert date here if you want it to appear below your name

%----------------------------------------------------------------------------------------

\begin{document}

\maketitle

%----------------------------------------------------------------------------------------
%	TABLE OF CONTENTS
%----------------------------------------------------------------------------------------

%\setcounter{tocdepth}{1} % Uncomment this line if you don't want subsections listed in the ToC

\newpage

\begin{homeworkProblem}
\begin{enumerate}
	\item Αν είναι δεδομένο ότι ένα αυτοκίνητο τρέχει με 150km/h τότε η πληροφορία είναι ότι τρέχει γρήγορα.
	\item Ο υπολογισμός του εμβαδού ενός κύκλου αποτελεί πρόβλημα βελτιστοποίησης.
	\item Ο υπολογισμός του εμβαδού του τραπεζίου, αποτελεί ένα δομημένο πρόβλημα.
	\item Η μετακίνηση από μία πόλη Α σε μία πόλη Β, με αυτοκίνητο, όταν υπάρχει μόνο ένας δρόμος είναι ένα αδόμητο πρόβλημα.
	\item Μία από τις πράξεις που μπορεί να εκτελέσει απευθείας ο υπολογιστής είναι ο πολλαπλασιασμός.
	\item Ο υπολογισμός της δευτεροβάθμιας εξίσωσης, αποτελεί πρόβλημα απόφασης.
	\item Κάθε αλγόριθμος πρέπει να πληρεί το κριτήριο της πληρότητας.
	\item Αλγόριθμος ο οποίος θα εμφανίζει τα αποτελέσματά του στην οθόνη δεν πληρεί το κριτήριο της εξόδου.
	\item Σε ένα διάγραμμα ροής, η εντολή διάβασε συμβολίζεται με ένα ορθογώνιο παραλληλόγραμμο.
	\item Αριστερά του τελεστή εκχώρησης ($\leftarrow$), μπορεί να υπάρξει μόνο μεταβλητή.
	\item Η λέξη διάβαζε μπορεί να αποτελέσει το όνομα μιας μεταβλητής.
	\item Ο πολλαπλασιασμός σε ένα πρόγραμμα γραμμένο σε ΓΛΩΣΣΑ έχει μεγαλύτερη προτεραιότητα σε σχέση με την ύψωση σε δύναμη.
	\item Η ΓΛΩΣΣΑ υποστηρίζει τρεις τύπους δεδομένων: τους ακέραιους, πραγματικούς και χαρακτήρες.
	\item Η λογική πράξη Η είναι αληθής όταν έστω μία από τις δύο προτάσεις που συνοδεύει είναι αληθής.
	\item Η λογική συνθήκη “α” < “β” είναι αληθής.
	\item Η λογική παράσταση x ΚΑΙ ΟΧΙ(x) είναι πάντα ψευδής ανεξάρτητα από την τιμή τιμή του x.
	\item Μία εντολή ΑΝ μπορεί προαιρετικά να συνοδεύεται από μία εντολή ΑΛΛΙΩΣ.
	\item Κάθε εντολή εμφωλευμένων ΑΝ μπορεί να μετατραπεί σε ισοδύναμες ΑΝ…ΑΛΛΙΩΣ\_ΑΝ….ΑΛΛΙΩΣ.
	\item Η εντολή ΑΡΧΗ\_ΕΠΑΝΑΛΗΨΗΣ…ΜΕΧΡΙΣ\_ΟΤΟΥ εκτελείται μέχρι η εντολή να γίνει αληθής.
	\item Το διάγραμμα ροής ενός αλγορίθμου που έχει μία εντολή ΑΝ…ΑΛΛΙΩΣ θα περιέχει δύο ρόμβους.
	\item Μία εντολή ΓΙΑ μπορεί να μετατραπεί σε ΟΣΟ ακόμα κι αν το βήμα της είναι αρνητικό.
	\item Κάθε εντολή ΟΣΟ μπορεί να μετατραπεί σε μια ισοδύναμη εντολή ΓΙΑ.
	\item Η εντολή ΓΙΑ μπορεί να χρησιμοποιηθεί για έλεγχο εγκυρότητας τιμής.
	\item Όταν το βήμα σε μια εντολή ΓΙΑ μειώνεται κατά ένα τότε αυτό μπορεί να παραληφθεί.
	\item Μία από τις λειτουργίες επί των δομών δεδομένων είναι η εισαγωγή ενός στοιχείου στην δομή.
	\item Ένα από τα πλεονεκτήματα των πινάκων είναι ότι χρειάζονται λίγο χώρο στην μνήμη.
	\item Για να προσδιορίσουμε την θέση σε έναν δισδιάστατο πίνακα χρειαζόμαστε μία μεταβλητή.
	\item Οι πίνακες είναι μία δομή δεδομένων που στηρίζεται στην τεχνική δυναμικής παραχώρησης μνήμης.
	\item Οι πίνακες καταναλώνουν μεγάλο χώρο στη μνήμη του υπολογιστή.
	\item Η σειριακή αναζήτηση θα πρέπει να αποφεύγεται όταν ο πίνακας είναι μεγάλος.
	\item H Fortran είναι μία γλώσσα, που ειδικεύεται στον προγραμματισμό εφαρμογών Τεχνητής Νοημοσύνης.
	\item H C είναι μία γλώσσα, που ειδικεύεται στον προγραμματισμό συστημάτων.
	\item Κατά τη μεταγλώττιση, το εκτελέσιμο πρόγραμμα παράγεται πριν το αντικείμενο.
	\item Το αποτέλεσμα του συνδέτη-φορτωτή είναι το πηγαίο πρόγραμμα.
	\item Η συγγραφή του πηγαίου προγράμματος και η διόρθωση των λαθών του γίνεται με την βοήθεια του συντάκτη.
	\item Τα λογικά λάθη μπορούν να εντοπιστούν από το διερμηνευτή αλλά δεν μπορούν στον μεταγλωττιστή.
	\item Οι γλώσσες μηχανής είναι στενά συνδεδεμένες με την αρχιτεκτονική του υπολογιστή.
	\item Σε μια συνάρτηση μπορούμε να χρησιμοποιήσουμε την εντολή ΓΡΑΨΕ, ενώ σε μια διαδικασία όχι.
	\item Ο αριθμός των τυπικών παραμέτρων κατά την κλήση ενός υποπρογράμματος πρέπει να είναι ίδιος με τον αριθμό των πραγματικών παραμέτρων στην δήλωση ενός προγράμματος.
	\item Κάθε υποπρόγραμμα θα πρέπει να είναι ανεξάρτητο από άλλα υποπρογράμματα.
	\item Μία συνάρτηση μπορεί να δεχτεί μία μόνο παράμετρο.
	\item Μία συνάρτηση μπορεί να εμφανίσει στην οθόνη την τιμή που υπολόγισε.
	\item Μία διαδικασία επιστρέφει στο πρόγραμμα που την έχει καλέσει τις τιμές όλων των παραμέτρων της
\end{enumerate}
\end{homeworkProblem}

%----------------------------------------------------------------------------------------

\end{document}