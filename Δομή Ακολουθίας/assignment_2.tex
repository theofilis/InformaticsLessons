%%%%%%%%%%%%%%%%%%%%%%%%%%%%%%%%%%%%%%%%%
% Programming/Coding Assignment
% LaTeX Template
%
% This template has been downloaded from:
% http://www.latextemplates.com
%
% Original author:
% Ted Pavlic (http://www.tedpavlic.com)
%
% Note:
% The \lipsum[#] commands throughout this template generate dummy text
% to fill the template out. These commands should all be removed when 
% writing assignment content.
%
% This template uses a Perl script as an example snippet of code, most other
% languages are also usable. Configure them in the "CODE INCLUSION 
% CONFIGURATION" section.
%
%%%%%%%%%%%%%%%%%%%%%%%%%%%%%%%%%%%%%%%%%

%----------------------------------------------------------------------------------------
%	PACKAGES AND OTHER DOCUMENT CONFIGURATIONS
%----------------------------------------------------------------------------------------

\documentclass{article}

\usepackage{fancyhdr} % Required for custom headers
\usepackage{lastpage} % Required to determine the last page for the footer
\usepackage{extramarks} % Required for headers and footers
\usepackage[usenames,dvipsnames]{color} % Required for custom colors
\usepackage{graphicx} % Required to insert images
\usepackage{listings} % Required for insertion of code
\usepackage{courier} % Required for the courier font
\usepackage{lipsum} % Used for inserting dummy 'Lorem ipsum' text into the template
\usepackage{fontspec}
\setmainfont{PF Universal}

% Margins
\topmargin=-0.45in
\evensidemargin=0in
\oddsidemargin=0in
\textwidth=6.5in
\textheight=9.0in
\headsep=0.25in

\linespread{1.1} % Line spacing

% Set up the header and footer
\pagestyle{fancy}
\lhead{\hmwkAuthorName} % Top left header
\chead{\hmwkClass\ (\hmwkClassInstructor\ \hmwkClassTime): \hmwkTitle} % Top center head
\rhead{\firstxmark} % Top right header
\lfoot{\lastxmark} % Bottom left footer
\cfoot{} % Bottom center footer
\rfoot{Page\ \thepage\ of\ \protect\pageref{LastPage}} % Bottom right footer
\renewcommand\headrulewidth{0.4pt} % Size of the header rule
\renewcommand\footrulewidth{0.4pt} % Size of the footer rule

\setlength\parindent{0pt} % Removes all indentation from paragraphs

%----------------------------------------------------------------------------------------
%	CODE INCLUSION CONFIGURATION
%----------------------------------------------------------------------------------------

\definecolor{MyDarkGreen}{rgb}{0.0,0.4,0.0} % This is the color used for comments
\lstloadlanguages{Perl} % Load Perl syntax for listings, for a list of other languages supported see: ftp://ftp.tex.ac.uk/tex-archive/macros/latex/contrib/listings/listings.pdf
\lstset{language=Perl, % Use Perl in this example
        frame=single, % Single frame around code
        basicstyle=\small\ttfamily, % Use small true type font
        keywordstyle=[1]\color{Blue}\bf, % Perl functions bold and blue
        keywordstyle=[2]\color{Purple}, % Perl function arguments purple
        keywordstyle=[3]\color{Blue}\underbar, % Custom functions underlined and blue
        identifierstyle=, % Nothing special about identifiers                                         
        commentstyle=\usefont{T1}{pcr}{m}{sl}\color{MyDarkGreen}\small, % Comments small dark green courier font
        stringstyle=\color{Purple}, % Strings are purple
        showstringspaces=false, % Don't put marks in string spaces
        tabsize=5, % 5 spaces per tab
        %
        % Put standard Perl functions not included in the default language here
        morekeywords={rand},
        %
        % Put Perl function parameters here
        morekeywords=[2]{on, off, interp},
        %
        % Put user defined functions here
        morekeywords=[3]{test},
       	%
        morecomment=[l][\color{Blue}]{...}, % Line continuation (...) like blue comment
        numbers=left, % Line numbers on left
        firstnumber=1, % Line numbers start with line 1
        numberstyle=\tiny\color{Blue}, % Line numbers are blue and small
        stepnumber=5 % Line numbers go in steps of 5
}

% Creates a new command to include a perl script, the first parameter is the filename of the script (without .pl), the second parameter is the caption
\newcommand{\perlscript}[2]{
\begin{itemize}
\item[]\lstinputlisting[caption=#2,label=#1]{#1.pl}
\end{itemize}
}

%----------------------------------------------------------------------------------------
%	DOCUMENT STRUCTURE COMMANDS
%	Skip this unless you know what you're doing
%----------------------------------------------------------------------------------------

% Header and footer for when a page split occurs within a problem environment
\newcommand{\enterProblemHeader}[1]{
\nobreak\extramarks{#1}{#1 continued on next page\ldots}\nobreak
\nobreak\extramarks{#1 (continued)}{#1 continued on next page\ldots}\nobreak
}

% Header and footer for when a page split occurs between problem environments
\newcommand{\exitProblemHeader}[1]{
\nobreak\extramarks{#1 (continued)}{#1 continued on next page\ldots}\nobreak
\nobreak\extramarks{#1}{}\nobreak
}

\setcounter{secnumdepth}{0} % Removes default section numbers
\newcounter{homeworkProblemCounter} % Creates a counter to keep track of the number of problems

\newcommand{\homeworkProblemName}{}
\newenvironment{homeworkProblem}[1][Πρόβλημα \arabic{homeworkProblemCounter}]{ % Makes a new environment called homeworkProblem which takes 1 argument (custom name) but the default is "Problem #"
\stepcounter{homeworkProblemCounter} % Increase counter for number of problems
\renewcommand{\homeworkProblemName}{#1} % Assign \homeworkProblemName the name of the problem
\section{\homeworkProblemName} % Make a section in the document with the custom problem count
\enterProblemHeader{\homeworkProblemName} % Header and footer within the environment
}{
\exitProblemHeader{\homeworkProblemName} % Header and footer after the environment
}

\newcommand{\problemAnswer}[1]{ % Defines the problem answer command with the content as the only argument
\noindent\framebox[\columnwidth][c]{\begin{minipage}{0.98\columnwidth}#1\end{minipage}} % Makes the box around the problem answer and puts the content inside
}

\newcommand{\homeworkSectionName}{}
\newenvironment{homeworkSection}[1]{ % New environment for sections within homework problems, takes 1 argument - the name of the section
\renewcommand{\homeworkSectionName}{#1} % Assign \homeworkSectionName to the name of the section from the environment argument
\subsection{\homeworkSectionName} % Make a subsection with the custom name of the subsection
\enterProblemHeader{} % Header and footer within the environment
}{
\enterProblemHeader{} % Header and footer after the environment
}

%----------------------------------------------------------------------------------------
%	NAME AND CLASS SECTION
%----------------------------------------------------------------------------------------

\newcommand{\hmwkTitle}{Δομή Ακολουθίας\ \#1} % Assignment title
\newcommand{\hmwkDueDate}{Σάββατο,\ Σεπτέμβριος\ 28,\ 2013} % Due date
\newcommand{\hmwkClass}{ΑΝΕΠ} % Course/class
\newcommand{\hmwkClassTime}{18:00} % Class/lecture time
\newcommand{\hmwkClassInstructor}{Θεοφίλης} % Teacher/lecturer
\newcommand{\hmwkAuthorName}{Γεώργιος Θεοφίλης} % Your name

%----------------------------------------------------------------------------------------
%	TITLE PAGE
%----------------------------------------------------------------------------------------

\title{
\vspace{2in}
\textmd{\textbf{\hmwkClass:\ \hmwkTitle}}\\
\normalsize\vspace{0.1in}\small{Due\ on\ \hmwkDueDate}\\
\vspace{0.1in}\large{\textit{\hmwkClassInstructor\ \hmwkClassTime}}
\vspace{3in}
}

\author{\textbf{\hmwkAuthorName}}
\date{} % Insert date here if you want it to appear below your name

%----------------------------------------------------------------------------------------

\begin{document}

\maketitle

%----------------------------------------------------------------------------------------
%	TABLE OF CONTENTS
%----------------------------------------------------------------------------------------

%\setcounter{tocdepth}{1} % Uncomment this line if you don't want subsections listed in the ToC

\newpage

\begin{homeworkProblem}
Να γραφεί αλγόριθμος και το σχετικό διάγραμμα ροής που θα διαβάζει δύο αριθμούς από το πληκτρολόγιο και θα υπολογίζει και εμφανίζει το άθροισμά τους.
\end{homeworkProblem}

\begin{homeworkProblem}
Να γραφεί αλγόριθμος και το σχετικό διάγραμμα ροής που θα διαβάζει το μήκος των πλευρών ενός ορθογωνίου από το πληκτρολόγιο και θα υπολογίζει και θα εμφανίζει το εμβαδό αυτού.
\end{homeworkProblem}

\begin{homeworkProblem}
Η συνολική αντίσταση $R$ δύο αντιστάσεων $R_1$ και $R_2$ συνδεδεμένων σε σειρά είναι $R_1 + R_2$ και παράλληλα $(R_1\cdot R_2)/(R_1+R_2)$ αντίστοιχα. Δεδομένων των τιμών R1 και R2, να γραφεί αλγόριθμος και το σχετικό διάγραμμα ροής που θα υπολογίζει και εμφανίζει τη συνολική αντίσταση R και με τους δύο τρόπους.
\end{homeworkProblem}

\begin{homeworkProblem}
Να γραφεί αλγόριθμος και το σχετικό διάγραμμα ροής που θα διαβάζει την τιμή ενός προϊόντος χωρίς ΦΠΑ και θα υπολογίζει και εμφανίζει την τελική του αξία, μαζί με τον ΦΠΑ (23%).
\end{homeworkProblem}

\begin{homeworkProblem}
Η Beta Bank δίνει $5\%$ ετήσιο επιτόκιο για τις καταθέσεις της. Να γίνει αλγόριθμος και το σχετικό διάγραμμα ροής που θα διαβάζει το ποσό ενός καταθέτη και θα εμφανίζει το ποσό που αυτός θα έχει μετά από 5 χρόνια.
\end{homeworkProblem}

\begin{homeworkProblem}
Να γίνει πρόγραμμα που θα δέχεται μία τιμή x και θα υπολογίζει την τιμή της παράστασης $4\sinh(x) + 9\sinh(x + 2)$, όπου το $\sinh(x)$ είναι το υπερβολικό ημίτονο του $x$ και ορίζεται $\sinh(x) = (e^x – e^{-x})/2$.
\end{homeworkProblem}

\begin{homeworkProblem}
Να γίνει αλγόριθμος, και το σχετικό διάγραμμα ροής που θα διαβάζει τον μισθό ενός υπαλλήλου και θα υπολογίζει από πόσα χαρτονομίσματα των 500, 200, 100, 50, 20, 10 και 5 ευρώ θα πρέπει να πληρωθεί. Ο αριθμός των χαρτονομισμάτων θα πρέπει να είναι ο λιγότερος δυνατός.
\end{homeworkProblem}

\begin{homeworkProblem}
Οι ηλεκτρονικοί υπολογιστές υπολογίζουν την τρέχουσα ημερομηνία, με βάση τον αριθμό των δευτερολέπτων που έχουν περάσει από την 1η Ιανουαρίου 1970. Να γραφεί αλγόριθμος που θα υπολογίζει και εμφανίζει την σημερινή ημερομηνία, διαβάζοντας από το πληκτρολόγιο τον αριθμό των δευτερολέπτων που πέρασαν από την 1/1/1970. Θεωρήστε ότι κάθε μήνας έχει 30 μέρες και ότι δεν υπάρχουν δίσεκτα έτη.
\end{homeworkProblem}

\begin{homeworkProblem}
Ένα κατάστημα ηλεκτρικών ειδών προσφέρει τα προϊόντα του με την εξής πολιτική: 30\% προκαταβολή, και το υπόλοιπο ποσό σε 36 άτοκες μηνιαίες δόσεις. Να γίνει αλγόριθμος που θα διαβάζει το ποσό αγοράς ενός πελάτη και θα υπολογίζει το ποσό της προκαταβολής και το ποσό κάθε δόσης.
\end{homeworkProblem}

\begin{homeworkProblem}
Μια εταιρεία κινητής τηλεφωνίας χρεώνει την αποστολή sms προς 0.07€. Στην τιμή αυτή δεν συμπεριλαμβάνεται ο ΦΠΑ (23\%). Η εταιρεία αποφάσισε για τον τρέχοντα μήνα να κάνει έκπτωση, στην τελική τιμή των μηνυμάτων, της 15\%. Να γίνει αλγόριθμος, που θα διαβάζει τον αριθμό των sms που έστειλε κάποιος συνδρομητής και θα εμφανίζει το ποσό που πρέπει να πληρώσει, λαμβάνοντας υπ” όψη τον ΦΠΑ και την έκπτωση που προσφέρει η εταιρεία.
\end{homeworkProblem}

\begin{homeworkProblem}
Το 1965 ο, συνιδρυτής της Intel, Gordon Moore διατύπωσε τον γνωστό σε όλους πια «νόμο του Moore» σύμφωνα με τον οποίο η χωρητικότητα των επεξεργαστών σε transistors (συνεπώς και η ταχύτητά τους) διπλασιάζεται κάθε 18 μήνες. Να γίνει αλγόριθμος που θα υπολογίζει και θα εμφανίζει τον αριθμό των transistors που θα περιέχει ένας επεξεργαστής σε 6 χρόνια από τώρα, αν ο σημερινός έχει 2000000000 (2 δισεκατομμύρια).
\end{homeworkProblem}

\begin{homeworkProblem}
Ένα πλήθος αυτοκινήτων λαμβάνει μέρος σε αγώνες ταχύτητας. Δεδομένου ότι στο τέλος τερματίζουν όλα τα αυτοκίνητα να γίνει αλγόριθμος που θα ζητά από το χρήστη 1) την κατανάλωση βενζίνης σε λίτρα ανά χιλιόμετρο του αυτοκινήτου, 2) το μήκος της πίστας σε χιλιόμετρα, 3) τον αριθμό των αυτοκινήτων που παίρνουν μέρος, 4) τoν αριθμό των γύρων, και θα υπολογίζει το σύνολο των καυσίμων που καταναλώθηκαν (σε λίτρα):
\begin{enumerate}
	\item για κάθε αμάξι σε ένα γύρο,
	\item από όλα τα αμάξια για σε ένα γύρο και
	\item από όλα τα αμάξια ανά αγώνα.
\end{enumerate}
\end{homeworkProblem}

%----------------------------------------------------------------------------------------

\end{document}