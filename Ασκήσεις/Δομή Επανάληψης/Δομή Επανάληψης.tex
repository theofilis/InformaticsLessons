%%%%%%%%%%%%%%%%%%%%%%%%%%%%%%%%%%%%%%%%%
% Programming/Coding Assignment
% LaTeX Template
%
% This template has been downloaded from:
% http://www.latextemplates.com
%
% Original author:
% Ted Pavlic (http://www.tedpavlic.com)
%
% Note:
% The \lipsum[#] commands throughout this template generate dummy text
% to fill the template out. These commands should all be removed when 
% writing assignment content.
%
% This template uses a Perl script as an example snippet of code, most other
% languages are also usable. Configure them in the "CODE INCLUSION 
% CONFIGURATION" section.
%
%%%%%%%%%%%%%%%%%%%%%%%%%%%%%%%%%%%%%%%%%

%----------------------------------------------------------------------------------------
%	PACKAGES AND OTHER DOCUMENT CONFIGURATIONS
%----------------------------------------------------------------------------------------

\documentclass{article}

\usepackage{fancyhdr} % Required for custom headers
\usepackage{lastpage} % Required to determine the last page for the footer
\usepackage{extramarks} % Required for headers and footers
\usepackage[usenames,dvipsnames]{color} % Required for custom colors
\usepackage{graphicx} % Required to insert images
\usepackage{listings} % Required for insertion of code
\usepackage{courier} % Required for the courier font
\usepackage{lipsum} % Used for inserting dummy 'Lorem ipsum' text into the template
\usepackage{fontspec}
\usepackage{float}
\setmainfont{PF Universal}

% Margins
\topmargin=-0.45in
\evensidemargin=0in
\oddsidemargin=0in
\textwidth=6.5in
\textheight=9.0in
\headsep=0.25in

\linespread{1.1} % Line spacing

% Set up the header and footer
\pagestyle{fancy}
\lhead{\hmwkAuthorName} % Top left header
\chead{\hmwkClass\ (\hmwkClassInstructor\ \hmwkClassTime): \hmwkTitle} % Top center head
\rhead{\firstxmark} % Top right header
\lfoot{\lastxmark} % Bottom left footer
\cfoot{} % Bottom center footer
\rfoot{Page\ \thepage\ of\ \protect\pageref{LastPage}} % Bottom right footer
\renewcommand\headrulewidth{0.4pt} % Size of the header rule
\renewcommand\footrulewidth{0.4pt} % Size of the footer rule

\setlength\parindent{0pt} % Removes all indentation from paragraphs

%----------------------------------------------------------------------------------------
%	CODE INCLUSION CONFIGURATION
%----------------------------------------------------------------------------------------

\definecolor{MyDarkGreen}{rgb}{0.0,0.4,0.0} % This is the color used for comments
\lstloadlanguages{Perl} % Load Perl syntax for listings, for a list of other languages supported see: ftp://ftp.tex.ac.uk/tex-archive/macros/latex/contrib/listings/listings.pdf
\lstset{language=Perl, % Use Perl in this example
        frame=single, % Single frame around code
        basicstyle=\small\ttfamily, % Use small true type font
        keywordstyle=[1]\color{Blue}\bf, % Perl functions bold and blue
        keywordstyle=[2]\color{Purple}, % Perl function arguments purple
        keywordstyle=[3]\color{Blue}\underbar, % Custom functions underlined and blue
        identifierstyle=, % Nothing special about identifiers                                         
        commentstyle=\usefont{T1}{pcr}{m}{sl}\color{MyDarkGreen}\small, % Comments small dark green courier font
        stringstyle=\color{Purple}, % Strings are purple
        showstringspaces=false, % Don't put marks in string spaces
        tabsize=5, % 5 spaces per tab
        %
        % Put standard Perl functions not included in the default language here
        morekeywords={rand},
        %
        % Put Perl function parameters here
        morekeywords=[2]{on, off, interp},
        %
        % Put user defined functions here
        morekeywords=[3]{test},
       	%
        morecomment=[l][\color{Blue}]{...}, % Line continuation (...) like blue comment
        numbers=left, % Line numbers on left
        firstnumber=1, % Line numbers start with line 1
        numberstyle=\tiny\color{Blue}, % Line numbers are blue and small
        stepnumber=5 % Line numbers go in steps of 5
}

% Creates a new command to include a perl script, the first parameter is the filename of the script (without .pl), the second parameter is the caption
\newcommand{\perlscript}[2]{
\begin{itemize}
\item[]\lstinputlisting[caption=#2,label=#1]{#1.pl}
\end{itemize}
}

%----------------------------------------------------------------------------------------
%	DOCUMENT STRUCTURE COMMANDS
%	Skip this unless you know what you're doing
%----------------------------------------------------------------------------------------

% Header and footer for when a page split occurs within a problem environment
\newcommand{\enterProblemHeader}[1]{
\nobreak\extramarks{#1}{#1 continued on next page\ldots}\nobreak
\nobreak\extramarks{#1 (continued)}{#1 continued on next page\ldots}\nobreak
}

% Header and footer for when a page split occurs between problem environments
\newcommand{\exitProblemHeader}[1]{
\nobreak\extramarks{#1 (continued)}{#1 continued on next page\ldots}\nobreak
\nobreak\extramarks{#1}{}\nobreak
}

\setcounter{secnumdepth}{0} % Removes default section numbers
\newcounter{homeworkProblemCounter} % Creates a counter to keep track of the number of problems

\newcommand{\homeworkProblemName}{}
\newenvironment{homeworkProblem}[1][Πρόβλημα \arabic{homeworkProblemCounter}]{ % Makes a new environment called homeworkProblem which takes 1 argument (custom name) but the default is "Problem #"
\stepcounter{homeworkProblemCounter} % Increase counter for number of problems
\renewcommand{\homeworkProblemName}{#1} % Assign \homeworkProblemName the name of the problem
\section{\homeworkProblemName} % Make a section in the document with the custom problem count
\enterProblemHeader{\homeworkProblemName} % Header and footer within the environment
}{
\exitProblemHeader{\homeworkProblemName} % Header and footer after the environment
}

\newcommand{\problemAnswer}[1]{ % Defines the problem answer command with the content as the only argument
\noindent\framebox[\columnwidth][c]{\begin{minipage}{0.98\columnwidth}#1\end{minipage}} % Makes the box around the problem answer and puts the content inside
}

\newcommand{\homeworkSectionName}{}
\newenvironment{homeworkSection}[1]{ % New environment for sections within homework problems, takes 1 argument - the name of the section
\renewcommand{\homeworkSectionName}{#1} % Assign \homeworkSectionName to the name of the section from the environment argument
\subsection{\homeworkSectionName} % Make a subsection with the custom name of the subsection
\enterProblemHeader{} % Header and footer within the environment
}{
\enterProblemHeader{} % Header and footer after the environment
}

%----------------------------------------------------------------------------------------
%	NAME AND CLASS SECTION
%----------------------------------------------------------------------------------------

\newcommand{\hmwkTitle}{Δομή Επανάληψης\ \#3} % Assignment title
\newcommand{\hmwkDueDate}{Σάββατο,\ Σεπτέμβριος\ 28,\ 2013} % Due date
\newcommand{\hmwkClass}{ΑΝΕΠ} % Course/class
\newcommand{\hmwkClassTime}{18:00} % Class/lecture time
\newcommand{\hmwkClassInstructor}{Θεοφίλης} % Teacher/lecturer
\newcommand{\hmwkAuthorName}{Γεώργιος Θεοφίλης} % Your name

%----------------------------------------------------------------------------------------
%	TITLE PAGE
%----------------------------------------------------------------------------------------

\title{
\vspace{2in}
\textmd{\textbf{\hmwkClass:\ \hmwkTitle}}\\
\normalsize\vspace{0.1in}\small{Due\ on\ \hmwkDueDate}\\
\vspace{0.1in}\large{\textit{\hmwkClassInstructor\ \hmwkClassTime}}
\vspace{3in}
}

\author{\textbf{\hmwkAuthorName}}
\date{} % Insert date here if you want it to appear below your name

%----------------------------------------------------------------------------------------

\begin{document}

\maketitle

%----------------------------------------------------------------------------------------
%	TABLE OF CONTENTS
%----------------------------------------------------------------------------------------

%\setcounter{tocdepth}{1} % Uncomment this line if you don't want subsections listed in the ToC

\newpage

\begin{homeworkProblem}
Να γίνει αλγόριθμος που θα υπολογίζει το άθροισμα $S = \sum_{i=0}^{n} 3i$ όπου το n θα δίνεται ως είσοδος από τον χρήστη.
\end{homeworkProblem}

\begin{homeworkProblem}
Να γίνει αλγόριθμος που θα υπολογίζει το άθροισμα $S = 1 + 2 + 3 + 4 + … + 300$. Ο αλγόριθμος θα πρέπει να εμφανίζει το άθροισμα κάθε φορά που προσθέτει 20 όρους. Δηλαδή, θα πρέπει να το εμφανίζει όταν φτάσει έως το 20, ύστερα έως το 40, μετά ως το 60 κ.ο.κ.
\end{homeworkProblem}

\begin{homeworkProblem}
Να γίνει αλγόριθμος που θα διαβάζει $n$ αριθμούς (θεωρείστε ότι $n > 3$) και θα εμφανίζει τους τρεις μεγαλύτερους.
\end{homeworkProblem}

\begin{homeworkProblem}
Να γίνει αλγόριθμος που θα εμφανίζει όλα τα ζεύγη $x, y$ για τα οποία ισχύει $x^3 – y = 7$. Οι αριθμοί x και y ανήκουν στο διάστημα $[-300, 300]$.
\end{homeworkProblem}

\begin{homeworkProblem}
Να γίνει αλγόριθμος που με δεδομένο έναν αριθμό $x$ (ο οποίος θα ανήκει στο διάστημα $[0, 1000]$) θα ζητά από τον χρήστη να τον μαντέψει. Ο αλγόριθμος θα σταματά όταν ο χρήστης βρει τον αριθμό ή ξεπεράσει τις $15$ προσπάθειες. Σε κάθε προσπάθεια ο αλγόριθμος θα πρέπει να ενημερώνει τον χρήστη, αν ο αριθμός που δόθηκε είναι μεγαλύτερος ή μικρότερος από το $x$.
\end{homeworkProblem}

\begin{homeworkProblem}
Να γίνει αλγόριθμος που θα διαβάζει το όνομα και τις βαθμολογίες σε 10 μαθήματα 100 μαθητών της Γ Λυκείου και θα εμφανίζει το όνομα εκείνου με τον καλύτερο μέσο όρο. Ο αλγόριθμος να μην επιτρέπει εισαγωγή βαθμού μεγαλύτερη από 20 και μικρότερη από 0.
\end{homeworkProblem}

\begin{homeworkProblem}
Να γίνει αλγόριθμος που θα μετρά το πλήθος των όρων του αθροίσματος $S = 1 + 3 – 9 + 27 – 81 + \dots$. ώστε το $S$ να μην ξεπεράσει το $4000$.
\end{homeworkProblem}

\begin{homeworkProblem}
Σε ένα παιχνίδι με τράπουλα η μέτρηση των πόντων γίνεται ως εξής: 1 πόντος για κάθε φιγούρα και για κάθε άσσο, 1 πόντος για κάθε δεκάρι εκτός του Δέκα καρώ που αξίζει 2, 1 πόντος για το 2 σπαθί. Να γίνει αλγόριθμος που, επαναληπτικά, θα διαβάζει τις κάρτες που έχουν οι παίκτες στην κατοχή τους και θα εμφανίζει το σύνολο των πόντων που έχουνε κερδίσει. Κάθε κάρτα αναπαρίσταται με δύο σύμβολα: τον αριθμό ή φιγούρα («A», «2», «3», «4», «5», «6», «7», «8», «9», «10», «J», «Q», «K») και το σύμβολο («κούπες», «καρό», «σπαθιά», «μπαστούνια»). Ο αλγόριθμος να σταματά όταν δώσουμε ένα μη αποδεκτό αριθμό ή σύμβολο.
\end{homeworkProblem}

\begin{homeworkProblem}
Ένα πολυκατάστημα δίνει τη δυνατότητα στους πελάτες του να αποπληρώσουν τις αγορές τους με δόσεις. Ο αριθμός των δόσεων εξαρτάται από το ύψος των αγορών. Έτσι αν κάποιος αγοράσει αντικείμενα αξίας έως 300 ευρώ μπορεί να αποπληρώσει το ποσό σε 3 έως 6 δόσεις. Αν το ποσό είναι πάνω από 300 έως 800 ευρώ τότε οι δόσεις είναι από 6 έως 9 και τέλος για περισσότερα από 800 ευρώ οι δόσεις αυξάνονται σε 9 έως 12. Να γίνει αλγόριθμος που θα διαβάζει το ποσό αποπληρωμής και να πληροφορεί τον χρήστη για τον αριθμό των δόσεων που μπορεί να έχει. Στη συνέχεια θα του ζητάει τον αριθμό των δόσεων που επιθυμεί (και να τον ζητάει συνεχόμενα μέχρι αυτός να είναι στα αποδεκτά όρια) και να εμφανίζει το ύψος της κάθε δόσης.
\end{homeworkProblem}

\begin{homeworkProblem}
Σύμφωνα με απόφαση του Υπουργείου Οικονομικών οι ιδιοκτήτες αυτοκινήτων από 0 έως 786 κ.ε. θα πληρώσουν για τέλη κυκλοφορίας 0 ευρώ, για αυτοκίνητα από 786 έως 1.357 κ.ε. 112 ευρώ, για αυτοκίνητα από 1.358 έως 1.928 κ.ε. 202 ευρώ, για αυτοκίνητα από 1.929 έως 2.357 κ.ε. 446 ευρώ και για αυτοκίνητα άνω των 2.358 κ.ε. 580 ευρώ. Να γίνει αλγόριθμος που θα διαβάζει επαναληπτικά τα κυβικά εκατοστά ενός αυτοκινήτου και θα τυπώνει το ποσό πληρωμής. Ο αλγόριθμος θα τερματίζει όταν εισαχθεί αρνητικός αριθμός. Στο τέλος να εμφανίζει τις συνολικές εισπράξεις που έγιναν.
\end{homeworkProblem}

\begin{homeworkProblem}
Η διαφορά ώρας ανάμεσα στην Ελλάδα και την Ινδία είναι 3 ώρες και 30 λεπτά. Αυτό σημαίνει πως όταν στην Ελλάδα η ώρα είναι 17.00 στην Ινδία είναι 20.30. Να γίνει αλγόριθμος που θα διαβάζει σε δύο μεταβλητές (μία για την ώρα και μία για τα λεπτά) την ώρα της Ελλάδας, σε 24ώρη μορφή, και θα εμφανίζει την ώρα της Ινδίας. π.χ. Ώρα Ελλάδας: 23.45, Ώρα Ινδίας: 3.15
\end{homeworkProblem}

\begin{homeworkProblem}
Να γίνει αλγόριθμος που θα διαβάζει 100 αριθμούς και θα εμφανίζει το πλήθος αυτών που είναι θετικοί άρτιοι, θετικοί περιττοί, αρνητικοί άρτιοι, αρνητικοί περιττοί και μηδέν.
\end{homeworkProblem}

\begin{homeworkProblem}
Να κάνετε αλγόριθμο που θα υπολογίζει την παράσταση $S = 5 – 2 + 10 – 4 + 15 – 6 + … +5N – 2N$ όπου το $N$ θα δίνεται από τον χρήστη.
\end{homeworkProblem}

\begin{homeworkProblem}
Να κάνετε αλγόριθμο που θα διαβάζει Ν αριθμούς (το Ν θα δίνεται επίσης από τον χρήστη) και θα τους αφαιρεί από μία αρχική τιμή. Έστω ότι η αρχική τιμή είναι το 200.
\end{homeworkProblem}

\begin{homeworkProblem}
Να κάνετε αλγόριθμο που θα διαβάζει 150 αριθμούς και θα εμφανίζει το ποσοστό των άρτιων αριθμών, το ποσοστών των περιττών, το ποσοστό αυτών που είναι μεγαλύτεροι από 75 και αυτών που είναι μικρότεροι από το 75.
\end{homeworkProblem}

\begin{homeworkProblem}
Να γίνει αλγόριθμος, που θα διαβάζει 30 θετικούς αριθμούς και θα βρίσκει τον μεγαλύτερο άρτιο και τον μεγαλύτερο περιττό.
\end{homeworkProblem}

\begin{homeworkProblem}
Να γίνει αλγόριθμος που θα διαβάζει τις τιμές πετρελαίου θέρμανσης 20 πρατηρίων, καθώς και την επωνυμία τους. Ο αλγόριθμος θα πρέπει να υπολογίζει και να εμφανίζει την επωνυμία του ακριβότερου και φθηνότερου πρατηρίου.
\end{homeworkProblem}

\begin{homeworkProblem}
Να γίνει αλγόριθμος που θα διαβάζει το ύψος σε εκατοστά 10 παικτών του μπάσκετ και θα εμφανίζει το ύψος του ψηλότερου και του κοντύτερου.
\end{homeworkProblem}

\begin{homeworkProblem}
Να αναπτυχθεί πρόγραμμα που θα διαβάζει άγνωστο πλήθος θετικών αριθμών και θα τερματίζει όταν εισαχθεί αρνητικός αριθμός ή μηδέν. Να εκτυπώνεται:
\begin{enumerate}
	\item Ο μεγαλύτερος αριθμός που διαβάστηκε
	\item Ο μικρότερος αριθμός που διαβάστηκε
	\item Το πλήθος των αριθμών που διαβάστηκαν
	\item Το πλήθος των άρτιων αριθμών που διαβάστηκαν
	\item Το πλήθος των περιττών αριθμών που διαβάστηκαν
	\item Ο μέσος όρος των στοιχείων που διαβάστηκαν
	\item Ο μέσος όρος των άρτιων αριθμών που διαβάστηκαν
	\item Ο μέσος όρος των περιττών αριθμών που διαβάστηκαν
\end{enumerate}
\end{homeworkProblem}

%----------------------------------------------------------------------------------------

\end{document}