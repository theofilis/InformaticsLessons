%%%%%%%%%%%%%%%%%%%%%%%%%%%%%%%%%%%%%%%%%
% Programming/Coding Assignment
% LaTeX Template
%
% This template has been downloaded from:
% http://www.latextemplates.com
%
% Original author:
% Ted Pavlic (http://www.tedpavlic.com)
%
% Note:
% The \lipsum[#] commands throughout this template generate dummy text
% to fill the template out. These commands should all be removed when 
% writing assignment content.
%
% This template uses a Perl script as an example snippet of code, most other
% languages are also usable. Configure them in the "CODE INCLUSION 
% CONFIGURATION" section.
%
%%%%%%%%%%%%%%%%%%%%%%%%%%%%%%%%%%%%%%%%%

%----------------------------------------------------------------------------------------
%	PACKAGES AND OTHER DOCUMENT CONFIGURATIONS
%----------------------------------------------------------------------------------------

\documentclass{article}

\usepackage{fancyhdr} % Required for custom headers
\usepackage{lastpage} % Required to determine the last page for the footer
\usepackage{extramarks} % Required for headers and footers
\usepackage[usenames,dvipsnames]{color} % Required for custom colors
\usepackage{graphicx} % Required to insert images
\usepackage{listings} % Required for insertion of code
\usepackage{courier} % Required for the courier font
\usepackage{lipsum} % Used for inserting dummy 'Lorem ipsum' text into the template
\usepackage{fontspec}
\setmainfont{PF Universal}

% Margins
\topmargin=-0.45in
\evensidemargin=0in
\oddsidemargin=0in
\textwidth=6.5in
\textheight=9.0in
\headsep=0.25in

\linespread{1.1} % Line spacing

% Set up the header and footer
\pagestyle{fancy}
\lhead{\hmwkAuthorName} % Top left header
\chead{\hmwkClass\ (\hmwkClassInstructor\ \hmwkClassTime): \hmwkTitle} % Top center head
\rhead{\firstxmark} % Top right header
\lfoot{\lastxmark} % Bottom left footer
\cfoot{} % Bottom center footer
\rfoot{Page\ \thepage\ of\ \protect\pageref{LastPage}} % Bottom right footer
\renewcommand\headrulewidth{0.4pt} % Size of the header rule
\renewcommand\footrulewidth{0.4pt} % Size of the footer rule

\setlength\parindent{0pt} % Removes all indentation from paragraphs

% Creates a new command to include a perl script, the first parameter is the filename of the script (without .pl), the second parameter is the caption
\newcommand{\perlscript}[2]{
\begin{itemize}
\item[]\lstinputlisting[caption=#2,label=#1]{#1.pl}
\end{itemize}
}

%----------------------------------------------------------------------------------------
%	DOCUMENT STRUCTURE COMMANDS
%	Skip this unless you know what you're doing
%----------------------------------------------------------------------------------------

% Header and footer for when a page split occurs within a problem environment
\newcommand{\enterProblemHeader}[1]{
\nobreak\extramarks{#1}{#1 continued on next page\ldots}\nobreak
\nobreak\extramarks{#1 (continued)}{#1 continued on next page\ldots}\nobreak
}

% Header and footer for when a page split occurs between problem environments
\newcommand{\exitProblemHeader}[1]{
\nobreak\extramarks{#1 (continued)}{#1 continued on next page\ldots}\nobreak
\nobreak\extramarks{#1}{}\nobreak
}

\setcounter{secnumdepth}{0} % Removes default section numbers
\newcounter{homeworkProblemCounter} % Creates a counter to keep track of the number of problems

\newcommand{\homeworkProblemName}{}
\newenvironment{homeworkProblem}[1][Παράδειγμα \arabic{homeworkProblemCounter}]{ % Makes a new environment called homeworkProblem which takes 1 argument (custom name) but the default is "Problem #"
\stepcounter{homeworkProblemCounter} % Increase counter for number of problems
\renewcommand{\homeworkProblemName}{#1} % Assign \homeworkProblemName the name of the problem
\section{\homeworkProblemName} % Make a section in the document with the custom problem count
\enterProblemHeader{\homeworkProblemName} % Header and footer within the environment
}{
\exitProblemHeader{\homeworkProblemName} % Header and footer after the environment
}

\newcommand{\problemAnswer}[1]{ % Defines the problem answer command with the content as the only argument
\noindent\framebox[\columnwidth][c]{\begin{minipage}{0.98\columnwidth}#1\end{minipage}} % Makes the box around the problem answer and puts the content inside
}

\newcommand{\homeworkSectionName}{}
\newenvironment{homeworkSection}[1]{ % New environment for sections within homework problems, takes 1 argument - the name of the section
\renewcommand{\homeworkSectionName}{#1} % Assign \homeworkSectionName to the name of the section from the environment argument
\subsection{\homeworkSectionName} % Make a subsection with the custom name of the subsection
\enterProblemHeader{} % Header and footer within the environment
}{
\enterProblemHeader{} % Header and footer after the environment
}

%----------------------------------------------------------------------------------------
%	NAME AND CLASS SECTION
%----------------------------------------------------------------------------------------

\newcommand{\hmwkTitle}{Σημειώσεις Δομή Ακολουθίας\ \#1} % Assignment title
\newcommand{\hmwkDueDate}{Σάββατο,\ Σεπτέμβριος\ 28,\ 2013} % Due date
\newcommand{\hmwkClass}{ΑΝΕΠ} % Course/class
\newcommand{\hmwkClassTime}{18:00} % Class/lecture time
\newcommand{\hmwkClassInstructor}{Θεοφίλης} % Teacher/lecturer
\newcommand{\hmwkAuthorName}{Γεώργιος Θεοφίλης} % Your name

%----------------------------------------------------------------------------------------
%	TITLE PAGE
%----------------------------------------------------------------------------------------

\title{
\vspace{2in}
\textmd{\textbf{\hmwkClass:\ \hmwkTitle}}\\
\normalsize\vspace{0.1in}\small{Due\ on\ \hmwkDueDate}\\
\vspace{0.1in}\large{\textit{\hmwkClassInstructor\ \hmwkClassTime}}
\vspace{3in}
}

\author{\textbf{\hmwkAuthorName}}
\date{} % Insert date here if you want it to appear below your name

%----------------------------------------------------------------------------------------

\begin{document}

\maketitle

%----------------------------------------------------------------------------------------
%	TABLE OF CONTENTS
%----------------------------------------------------------------------------------------

%\setcounter{tocdepth}{1} % Uncomment this line if you don't want subsections listed in the ToC

\newpage

%----------------------------------------------------------------------------------------
\begin{homeworkProblem}
Έστω ότι σε ένα δελτίο στοιχήματος μπορούν να συμπληρωθούν τρεις ακριβώς αγώνες. Ένα δελτίο κερδίζει αν προβλεφθούν ορθά και οι τρεις αγώνες. Το ποσό που κερδίζεται είναι ίσο με το ποσό που ποντάρει ο παίχτης επί τις αποδόσεις των τριών αγώνων. \\ \\
Για παράδειγμα, αν ένας παίχτης παίξει τρεις αγώνες με απόδοση 1.5, 2.3 και 1.4 με 3 € το ποσό που κερδίζει εάν τους προβλέψει όλους είναι ίσο με 1.5 x 2.3 x 1.4 x 3.
Να γίνει αλγόριθμος που θα ζητάει τις τρεις αποδόσεις, το ποσό που ποντάρει ο παίχτης και θα δίνει σαν αποτέλεσμα πόσα χρήματα μπορεί να κερδίσει.

\problemAnswer{
\textbf{Αλγόριθμος} Στοίχημα \\
\ 	\textbf{Διάβασε} ποσό, απόδοση1, απόδοση2, απόδοση3 \\
πιθανό\_ποσό\_νίκης $\leftarrow$ ποσό * απόδοση1 * απόδοση2 * απόδοση3 \\
\textbf{Εμφάνισε} πιθανό\_ποσό\_νίκης \\
\textbf{Τέλος} Στοίχημα \\
}
\end{homeworkProblem}

\begin{homeworkProblem}
Σε έναν φιλικό ποδοσφαιρικό αγώνα ισχύει γενικό εισιτήριο αξίας 15 ευρώ. Από τις εισπράξεις του αγώνα το 20\% κρατείται από την εφορία. \\ \\Από αυτά που περισσεύουν, το 75\% κρατάει η γηπεδούχος ομάδα, ενώ το 25\% η φιλοξενούμενη. Να γίνει αλγόριθμος που θα διαβάζει τον αριθμό των εισιτηρίων που αγοράστηκαν και θα υπολογίζει και εμφανίζει, τις εισπράξεις του αγώνα, το ποσό που κρατάει η εφορία, το ποσό που προορίζεται για την γηπεδούχο ομάδα και το ποσό για την φιλοξενούμενη,

\problemAnswer{}
\end{homeworkProblem}

\begin{homeworkProblem}
Ένα super market, προσφέρει στους πελάτες του την δυνατότητα συλλογής πόντων στις αγορές που πραγματοποιούν. Έτσι για κάθε 30 ευρώ αγοράς κερδίζουν 1 πόντο. Για κάθε 10 πόντους που έχουν στην συλλογή τους, έχουν το δικαίωμα να τους εξαργυρώσουν με μια δωροεπιταγή των 3 ευρώ. Να γίνει αλγόριθμος, που θα διαβάζει το συνολικό ποσό αγορών που ένας πελάτης πραγματοποίησε στο super market και να εμφανίζει το ποσό της δωροεπιταγής βάσει των πόντων που συνέλεξε.

\problemAnswer{}
\end{homeworkProblem}

\begin{homeworkProblem}
Ένας αυτόματος πωλητής αναψυκτικών λειτουργεί δεχόμενος μόνο κέρματα των 50, 20 και 10 λεπτών του ευρώ. Επίσης μπορεί να δώσει ρέστα του ίδιου ακριβώς τύπου (50, 20 και 10 λεπτών). Να γίνει αλγόριθμος που θα προσομοιώνει την λειτουργία του αυτόματου πωλητή: \\ \\ Αρχικά θα διαβάζει το αριθμό των αναψυκτικών που κάποιος επιθυμεί να αγοράσει. Στην συνέχεια θα διαβάζει τρεις τιμές, που αντιστοιχούν στον αριθμό των κερμάτων που το μηχάνημα μπορεί να δεχτεί. Στο τέλος, θα εμφανίζει τα ρέστα που πρέπει να επιστρέψει το μηχάνημα, αναλυτικά σε αριθμό 50λεπτων, 20λεπτων και 10λεπτων. Να σημειωθεί πως κάθε αναψυκτικό κοστίζει 60 λεπτά.

\problemAnswer{}
\end{homeworkProblem}

\begin{homeworkProblem}
Η χωρητικότητα των σκληρών δίσκων (τουλάχιστον μέχρι πριν μερικά χρόνια) μετριέται συνήθως σε Gigabytes (GB). Ένα Gigabyte, αποτελείται από 1024 Megabytes (ΜΒ). Ομοίως, ένα Megabyte αποτελείται από 1024 Kilobytes (ΚΒ) και ένα Kilobyte από 1024 Bytes. Να γίνει αλγόριθμος ο οποίος θα διαβάζει τη χωρητικότητα ενός σκληρού δίσκου σε GB και θα εκτυπώνει τον αριθμό των MB, KB και Bytes που είναι ισοδύναμο.

\problemAnswer{}
\end{homeworkProblem}
%----------------------------------------------------------------------------------------

\end{document}