%%%%%%%%%%%%%%%%%%%%%%%%%%%%%%%%%%%%%%%%%
% Programming/Coding Assignment
% LaTeX Template
%
% This template has been downloaded from:
% http://www.latextemplates.com
%
% Original author:
% Ted Pavlic (http://www.tedpavlic.com)
%
% Note:
% The \lipsum[#] commands throughout this template generate dummy text
% to fill the template out. These commands should all be removed when 
% writing assignment content.
%
% This template uses a Perl script as an example snippet of code, most other
% languages are also usable. Configure them in the "CODE INCLUSION 
% CONFIGURATION" section.
%
%%%%%%%%%%%%%%%%%%%%%%%%%%%%%%%%%%%%%%%%%

%----------------------------------------------------------------------------------------
%	PACKAGES AND OTHER DOCUMENT CONFIGURATIONS
%----------------------------------------------------------------------------------------

\documentclass{article}

\usepackage{fancyhdr} % Required for custom headers
\usepackage{lastpage} % Required to determine the last page for the footer
\usepackage{extramarks} % Required for headers and footers
\usepackage[usenames,dvipsnames]{color} % Required for custom colors
\usepackage{graphicx} % Required to insert images
\usepackage{listings} % Required for insertion of code
\usepackage{courier} % Required for the courier font
\usepackage{lipsum} % Used for inserting dummy 'Lorem ipsum' text into the template
\usepackage{fontspec}
\usepackage{float}
\setmainfont{PF Universal}

% Margins
\topmargin=-0.45in
\evensidemargin=0in
\oddsidemargin=0in
\textwidth=6.5in
\textheight=9.0in
\headsep=0.25in

\linespread{1.1} % Line spacing

% Set up the header and footer
\pagestyle{fancy}
\lhead{\hmwkAuthorName} % Top left header
\chead{\hmwkClass\ (\hmwkClassInstructor\ \hmwkClassTime): \hmwkTitle} % Top center head
\rhead{\firstxmark} % Top right header
\lfoot{\lastxmark} % Bottom left footer
\cfoot{} % Bottom center footer
\rfoot{Page\ \thepage\ of\ \protect\pageref{LastPage}} % Bottom right footer
\renewcommand\headrulewidth{0.4pt} % Size of the header rule
\renewcommand\footrulewidth{0.4pt} % Size of the footer rule

\setlength\parindent{0pt} % Removes all indentation from paragraphs

%----------------------------------------------------------------------------------------
%	CODE INCLUSION CONFIGURATION
%----------------------------------------------------------------------------------------

\definecolor{MyDarkGreen}{rgb}{0.0,0.4,0.0} % This is the color used for comments
\lstloadlanguages{Perl} % Load Perl syntax for listings, for a list of other languages supported see: ftp://ftp.tex.ac.uk/tex-archive/macros/latex/contrib/listings/listings.pdf
\lstset{language=Perl, % Use Perl in this example
        frame=single, % Single frame around code
        basicstyle=\small\ttfamily, % Use small true type font
        keywordstyle=[1]\color{Blue}\bf, % Perl functions bold and blue
        keywordstyle=[2]\color{Purple}, % Perl function arguments purple
        keywordstyle=[3]\color{Blue}\underbar, % Custom functions underlined and blue
        identifierstyle=, % Nothing special about identifiers                                         
        commentstyle=\usefont{T1}{pcr}{m}{sl}\color{MyDarkGreen}\small, % Comments small dark green courier font
        stringstyle=\color{Purple}, % Strings are purple
        showstringspaces=false, % Don't put marks in string spaces
        tabsize=5, % 5 spaces per tab
        %
        % Put standard Perl functions not included in the default language here
        morekeywords={rand},
        %
        % Put Perl function parameters here
        morekeywords=[2]{on, off, interp},
        %
        % Put user defined functions here
        morekeywords=[3]{test},
       	%
        morecomment=[l][\color{Blue}]{...}, % Line continuation (...) like blue comment
        numbers=left, % Line numbers on left
        firstnumber=1, % Line numbers start with line 1
        numberstyle=\tiny\color{Blue}, % Line numbers are blue and small
        stepnumber=5 % Line numbers go in steps of 5
}

% Creates a new command to include a perl script, the first parameter is the filename of the script (without .pl), the second parameter is the caption
\newcommand{\perlscript}[2]{
\begin{itemize}
\item[]\lstinputlisting[caption=#2,label=#1]{#1.pl}
\end{itemize}
}

%----------------------------------------------------------------------------------------
%	DOCUMENT STRUCTURE COMMANDS
%	Skip this unless you know what you're doing
%----------------------------------------------------------------------------------------

% Header and footer for when a page split occurs within a problem environment
\newcommand{\enterProblemHeader}[1]{
\nobreak\extramarks{#1}{#1 continued on next page\ldots}\nobreak
\nobreak\extramarks{#1 (continued)}{#1 continued on next page\ldots}\nobreak
}

% Header and footer for when a page split occurs between problem environments
\newcommand{\exitProblemHeader}[1]{
\nobreak\extramarks{#1 (continued)}{#1 continued on next page\ldots}\nobreak
\nobreak\extramarks{#1}{}\nobreak
}

\setcounter{secnumdepth}{0} % Removes default section numbers
\newcounter{homeworkProblemCounter} % Creates a counter to keep track of the number of problems

\newcommand{\homeworkProblemName}{}
\newenvironment{homeworkProblem}[1][Πρόβλημα \arabic{homeworkProblemCounter}]{ % Makes a new environment called homeworkProblem which takes 1 argument (custom name) but the default is "Problem #"
\stepcounter{homeworkProblemCounter} % Increase counter for number of problems
\renewcommand{\homeworkProblemName}{#1} % Assign \homeworkProblemName the name of the problem
\section{\homeworkProblemName} % Make a section in the document with the custom problem count
\enterProblemHeader{\homeworkProblemName} % Header and footer within the environment
}{
\exitProblemHeader{\homeworkProblemName} % Header and footer after the environment
}

\newcommand{\problemAnswer}[1]{ % Defines the problem answer command with the content as the only argument
\noindent\framebox[\columnwidth][c]{\begin{minipage}{0.98\columnwidth}#1\end{minipage}} % Makes the box around the problem answer and puts the content inside
}

\newcommand{\homeworkSectionName}{}
\newenvironment{homeworkSection}[1]{ % New environment for sections within homework problems, takes 1 argument - the name of the section
\renewcommand{\homeworkSectionName}{#1} % Assign \homeworkSectionName to the name of the section from the environment argument
\subsection{\homeworkSectionName} % Make a subsection with the custom name of the subsection
\enterProblemHeader{} % Header and footer within the environment
}{
\enterProblemHeader{} % Header and footer after the environment
}

%----------------------------------------------------------------------------------------
%	NAME AND CLASS SECTION
%----------------------------------------------------------------------------------------

\newcommand{\hmwkTitle}{Υποπρογράμματα\ \#6} % Assignment title
\newcommand{\hmwkDueDate}{Σάββατο,\ Σεπτέμβριος\ 28,\ 2013} % Due date
\newcommand{\hmwkClass}{ΑΝΕΠ} % Course/class
\newcommand{\hmwkClassTime}{18:00} % Class/lecture time
\newcommand{\hmwkClassInstructor}{Θεοφίλης} % Teacher/lecturer
\newcommand{\hmwkAuthorName}{Γεώργιος Θεοφίλης} % Your name

%----------------------------------------------------------------------------------------
%	TITLE PAGE
%----------------------------------------------------------------------------------------

\title{
\vspace{2in}
\textmd{\textbf{\hmwkClass:\ \hmwkTitle}}\\
\normalsize\vspace{0.1in}\small{Due\ on\ \hmwkDueDate}\\
\vspace{0.1in}\large{\textit{\hmwkClassInstructor\ \hmwkClassTime}}
\vspace{3in}
}

\author{\textbf{\hmwkAuthorName}}
\date{} % Insert date here if you want it to appear below your name

%----------------------------------------------------------------------------------------

\begin{document}

\maketitle

%----------------------------------------------------------------------------------------
%	TABLE OF CONTENTS
%----------------------------------------------------------------------------------------

%\setcounter{tocdepth}{1} % Uncomment this line if you don't want subsections listed in the ToC

\newpage

\begin{homeworkProblem}
Να φτιάξετε συνάρτηση με όνομα ΣΤΡΟΓΓΥΛΟΠΟΙΗΣΗ που θα δέχεται σαν παράμετρο έναν πραγματικό αριθμό και θα τον επιστρέφει στρογγυλοποιημένο ως προς τη μονάδα. Δηλαδή, όταν καλέσουμε την συνάρτηση ΣΤΡΟΓΓΥΛΟΠΟΙΗΣΗ(14,219) θα πρέπει να μας επιστρέφει 14 και όταν καλούμε ΣΤΡΟΓΓΥΛΟΠΟΙΗΣΗ(-9,679) θα επιστρέφει -10.
\end{homeworkProblem}

\begin{homeworkProblem}
Να κάνετε πρόγραμμα το οποίο αφού διαβάζει τα στοιχεία ενός πίνακα 10×10 ακεραίων αριθμών θα αναδιατάσσει τις σειρές του με αύξουσα σειρά ως προς το άθροισμα τους. Δηλαδή η σειρά με το μικρότερο άθροισμα να ανέβει πρώτη, μετά η δεύτερη κ.ο.κ
\end{homeworkProblem}

\begin{homeworkProblem}
Η βιβλιοθήκη του δήμου, θέλει να οργανώσει τα διαθέσιμα βιβλία της ηλεκτρονικά. Διαθέτει 3000 βιβλία τα οποία θα αποθηκεύσει στον πίνακα Β[3000]. Επιπρόσθετα, στον πίνακα Σ[3000] θα αποθηκεύσει το όνομα του Συγγραφέα. Η βιβλιοθήκη καταχωρεί σε έναν ακόμη πίνακα ΟΝ[1000] το ονοματεπώνυμο των δανειστών της. Ο δανεισμός των βιβλίων είναι μηνιαίος. Τέλος σε έναν πίνακα Δ[3000, 12] καταχωρούνται οι δανεισμοί κάθε ενός από τα 3000 βιβλία για τους 12 μήνες του χρόνου. Αν κάποιο βιβλίο έχει δανειστεί για κάποιο μήνα τότε στην αντίστοιχη θέση του πίνακα καταχωρείται ο αύξων αριθμός του δανειστή από τον πίνακα ΟΝ, αλλιώς, αν το βιβλίο δεν είναι δανεισμένο μπαίνει ο αριθμός 0.
\begin{enumerate}
	\item Να γίνει πρόγραμμα το οποίο:
	\begin{enumerate}
		\item Θα κάνει χρήση του υποπρογράμματος με όνομα ΕΙΣΟΔΟΣ το οποίο θα διαβάζει τους προαναφερθέντες πίνακες από το πληκτρολόγιο του χρήστη και θα τους επιστρέφει στο κύριο πρόγραμμα.
		\item Θα βρίσκει, και θα εμφανίζει για πιο βιβλίο (όνομα και συγγραφέας) έχουν γίνει οι περισσότεροι δανεισμοί. Αν είναι περισσότερα από έναν τότε να εμφανίζονται όλα.
		\item Θα εντοπίζει και θα εμφανίζει για κάθε μήνα, τον δανειστή που δανείστηκε τα περισσότερα βιβλία.
		\item Θα εντοπίζει και θα εμφανίζει τον πιο επιτυχημένο μήνα σε αριθμό δανεισμών.
		\item $(**)$ Θα διαβάζει το όνομα ενός συνδρομητή και να εμφανίζει, ποιον συγγραφέα συνήθως προτιμά ο συγκεκριμένος δανειστής.
	\end{enumerate}
	\item Να υλοποιήσετε το υποπρόγραμμα ΕΙΣΟΔΟΣ
\end{enumerate}
\end{homeworkProblem}

\begin{homeworkProblem}
Να γράψετε τη συνάρτηση ΝΙΟΣΤΟΣ\_ΑΡΙΘΜΟΣ που παίρνει σαν παράμετρο δύο ακέραιους x και n και θα επιστρέψει το n-οστό ψηφίο του x από το τέλος. Για παράδειγμα η κλήση της συνάρτησης ΝΙΟΣΤΟΣ\_ΑΡΙΘΜΟΣ(534345, 3) θα επιστρέφει 3.
\end{homeworkProblem}

\begin{homeworkProblem}
Μία εταιρία κινητής τηλεφωνίας, έχει 10.000 πελάτες. Για αυτούς έχει καταχωρημένα τα ονόματά τους σε έναν πίνακα ΠΕΛΑΤΗΣ[10000], τον χρόνο ομιλίας σε λεπτά για κάθε έναν, για τους 12 μήνες της χρονιάς που πέρασε στον πίνακα ΧΡΟΝΟΣ\_ΟΜΙΛΙΑΣ[10000, 12], τα μηνύματα που έχουνε στείλει για κάθε μήνα σε έναν πίνακα ΜΗΝΥΜΑΤΑ[10000, 12] και σε έναν πίνακα ΤΡΟΠΟΣ\_ΧΡΕΩΣΗΣ[10000] τον αριθμό 1 ή 2 ανάλογα με την τιμολογιακή πολιτική που επέλεξαν. Οι τιμολογιακές πολιτικές που υπάρχουν είναι:

\begin{table}[H]
    \begin{center}
    \begin{tabular}{lll}
    \textbf{Τιμολογιακή πολιτική}  & \textbf{1} & \textbf{2} \\
   	Πάγιο       & 10 € & 15 €\\
	Δωρεάν Χρόνος Ομιλίας      & 60 λεπτά &   100 λεπτά  \\
    	Δωρεάν Μηνύματα     &  60 & 100    \\
    	Χρέωση ανά λεπτό μετά τον δωρεάν χρόνο ομιλίας     & 0,15€ / λεπτό &  0,13€ / λεπτό\\
    \end{tabular}
    \end{center}
\end{table}

\begin{enumerate}
	\item Να γίνει διαδικασία με όνομα ΕΙΣΑΓΩΓΗ\_ΔΕΔΟΜΕΝΩΝ η οποία θα γεμίζει τους προηγούμενους πίνακες και θα ελέγχει την σωστή εισαγωγή δεδομένων για τον πίνακα ΤΡΟΠΟΣ\_ΧΡΕΩΣΗΣ (αποδεκτές τιμές μόνο οι 1 και 2).
	\item Να γίνει η συνάρτηση ΥΠΟΛΟΓΙΣΜΟΣ\_ΧΡΕΩΣΗΣ που θα υπολογίζει το ποσό που πρέπει να πληρώσει κάποιος βάσει του χρόνου ομιλίας του και του της τιμολογιακής πολιτικής που επέλεξε.
	\item Να γίνει η διαδικασία ΕΜΦΑΝΙΣΗ\_ΣΤΟΙΧΕΙΩΝ\_ΠΑΝΩ\_ΑΠΟ\_ΟΡΙΟ που θα εμφανίζει τους πελάτες της εταιρίας των οποίων ο λογαριασμός τους (μηνιαίος ή ετήσιος) είναι πάνω από ένα δοθέν όριο καθώς και το ποσό που πρέπει να πληρώσουν.
	\item Να γίνει το κυρίως πρόγραμμα το οποίο θα εντοπίζει τους πελάτες που έχουν μέσο μηνιαίο λογαριασμό πάνω από 150€ και στην συνέχεια τους πελάτες που έχουν ετήσιο λογαριασμό πάνω από 2000€.
\end{enumerate}
\end{homeworkProblem}


%----------------------------------------------------------------------------------------

\end{document}